\documentclass[10pt,landscape]{article}
\usepackage{multicol}
\usepackage{calc}
\usepackage{ifthen}
\usepackage[landscape]{geometry}
\usepackage{hyperref}
\usepackage{amsmath}
\usepackage{mathtools}

% To make this come out properly in landscape mode, do one of the following
% 1.
%  pdflatex latexsheet.tex
%
% 2.
%  latex latexsheet.tex
%  dvips -P pdf  -t landscape latexsheet.dvi
%  ps2pdf latexsheet.ps


% This sets page margins to .5 inch if using letter paper, and to 1cm
% if using A4 paper. (This probably isn't strictly necessary.)
% If using another size paper, use default 1cm margins.
\ifthenelse{\lengthtest { \paperwidth = 11in}}
	{ \geometry{top=.5in,left=.5in,right=.5in,bottom=.5in} }
	{\ifthenelse{ \lengthtest{ \paperwidth = 297mm}}
		{\geometry{top=1cm,left=1cm,right=1cm,bottom=1cm} }
		{\geometry{top=1cm,left=1cm,right=1cm,bottom=1cm} }
	}

% Turn off header and footer
\pagestyle{empty}

% Redefine section commands to use less space
\makeatletter
\renewcommand{\section}{\@startsection{section}{1}{0mm}%
                                {-1ex plus -.5ex minus -.2ex}%
                                {0.5ex plus .2ex}%x
                                {\normalfont\large\bfseries}}
\renewcommand{\subsection}{\@startsection{subsection}{2}{0mm}%
                                {-1explus -.5ex minus -.2ex}%
                                {0.5ex plus .2ex}%
                                {\normalfont\normalsize\bfseries}}
\renewcommand{\subsubsection}{\@startsection{subsubsection}{3}{0mm}%
                                {-1ex plus -.5ex minus -.2ex}%
                                {1ex plus .2ex}%
                                {\normalfont\small\bfseries}}
\makeatother

% Define BibTeX command
\def\BibTeX{{\rm B\kern-.05em{\sc i\kern-.025em b}\kern-.08em
    T\kern-.1667em\lower.7ex\hbox{E}\kern-.125emX}}

% Don't print section numbers
\setcounter{secnumdepth}{0}


\setlength{\parindent}{0pt}
\setlength{\parskip}{0pt plus 0.5ex}


% -----------------------------------------------------------------------

\begin{document}

\raggedright
\footnotesize
\begin{multicols}{3}


% multicol parameters
% These lengths are set only within the two main columns
%\setlength{\columnseprule}{0.25pt}
\setlength{\premulticols}{1pt}
\setlength{\postmulticols}{1pt}
\setlength{\multicolsep}{1pt}
\setlength{\columnsep}{2pt}

\begin{center}
     \Large{\textbf{ELEC412 Cheat Sheet}} \\
\end{center}


\section{Chapter 3: PN Junctions}
\subsection{3.2 - Concept of PN Junction}
\subsubsection{Energy Band Diagram for a PN Junction}
\begin{tabular}{@{}ll@{}}
Built in potential: $\Phi_{bi}=kTln(\frac{N_AN_D}{n^2_i})$ \textbf{(1)}\\
\end{tabular}


\subsubsection{I-V Characteristics of a PN Junction}
\begin{tabular}{@{}ll@{}}
$I=I_0exp(V_a/V_t)$ \textbf{(2)}\\
Dynamic resistance: $R_{dyn}=\frac{d(V_a)}{dI}=\frac{V_t}{I}$ \textbf{(3)}\\
IV characteristics of pn junc. w/ rev. current: \\
$I=I_0exp(V_a/V_t)-I_r$ \textbf{(4)}\\
If $I_r=I_0$ then $I=I_0[exp(V_a/V_t)-1]$ - ideal diode eqn.\textbf{(5)}\\
Hole diffusion current flowing into the n-side in a $p^+n$ junc.: \\
$I_p(x)=qD_p[d(p_n(x))/dx]A_{cs}$ \textbf{(6)}\\
$ p_n(x)=p^{\prime}exp(-x/L_p)+p_{n0}$ \textbf{(7)}\\
$ d(p_n(x))/dx=-(p_n(x)-p_{n0})/L_p$ \textbf{(8)}\\
$ I_p(x)=-qD_p(p_n(x)-p_{n0})A_{cs}/L_p$ \textbf{(9)}\\
$ p^{\prime}=p_{n0}[exp(qV_a/kT)-1]$ \textbf{(10)}\\
$ I_p=[qD_pn^2_iA_{cs}/(N_DL_p)][exp(qV_a/kT)-1]$ \textbf{(11)}\\
$ I=I_p+I_n=I_0[exp(V_a/V_t)]-1$ \textbf{(12)}\\
where $ I_0=qD_pn^2_iA_{cs}/(N_DL_p)+qD_nn^2_iA_{cs}/(N_AL_p)$ \\
$ I_G=qn_iW_jA_{cs}/(2\tau_0)$ \textbf{(13)}\\
$ I=I_0exp[(V_a/\eta_IV_t)-1]-I_G$ \textbf{(14)}\\
\end{tabular}


\subsubsection{Space-Charge and Charge Storage Effects}
\begin{tabular}{@{}ll@{}}
$ C_{jun}=\varepsilon_sA_{cs}/W_j$ \textbf{(15)}\\
$ W_j=\sqrt{2\varepsilon_s(V_{bi}-V_a)/(qN_{eff})}$ \textbf{(16)}\\
$ W_j=[12\varepsilon_s(V_{bi}-V_a)/(qa)]^{1/3}$ \textbf{(17)}\\
$ C_{diff}=[qA_{cs}(L_pp_{n0}+L_np_{p0})/2V_t]exp(V_0/V_t)$ \textbf{(18)}\\
$ C_{diff}\approx I_0\tau_p/(2V_t)$ \textbf{(19)}\\
\end{tabular}


\subsubsection{Tunnel PN Junctions}
\begin{tabular}{@{}ll@{}}
$ D_{12}=D_bexp(-2d_b\sqrt{2m^*_e(\Phi_B-E})/h^{\prime})$ \textbf{(20)}\\
$ I_{tun}=qA_{cs}\int_{E_c}^{\acute E}[f_tS_1D_{12}S_2]dE$ \textbf{(21)}\\
\end{tabular}


\subsubsection{Junction Breakdown}
\begin{tabular}{@{}ll@{}}
$ V_{br}=\epsilon_s \acute{E}^2_{cr}/(2qN_{eff})$ \textbf{(22)}\\
$ d(j_n)/dx-(\alpha_i-\beta_i)j_n=-(\alpha_i-\beta_i)j_T+\alpha_ij_T$ \textbf{(23)}\\
If $ \alpha_i=\beta_i$ then $ M_n=j_n(L)/j_n(0)=1/[1-\int_0^L(\alpha_i)dx]$ \textbf{(24)}\\
$ V_{br}\approx 4E_g/q$ \textbf{(25)}\\
\end{tabular}


\subsubsection{Noise in PN Junctions}
\begin{tabular}{@{}ll@{}}
$ \hat{i}^2=2q(I_d+2I_0)\delta f$ \textbf{(26)}\\
\end{tabular}


\subsubsection{Heterojunctions}
\begin{tabular}{@{}ll@{}}
$ \Delta E_c = \xi_1-\xi_2$ \textbf{(27)}\\
$ \phi_{bi}=E_{g1}-\Delta E_n-\Delta E_p+\Delta E_c$ \textbf{(28)}\\
$ W_j=W_n+W+p$ \textbf{(29)}\\
$ W_n=\sqrt{2\epsilon_1\epsilon_2N_AV_{bi}/[qN_D(\epsilon_2N_D+\epsilon_1N_A)]}$ \\
$ W_p=\sqrt{2\epsilon_1\epsilon_2N_DV_{bi}/[qN_A(\epsilon_2N_D+\epsilon_1N_A)]}$ \\
$ C_{jun}=\sqrt{\epsilon_1\epsilon_2N_DN_A/[2(\epsilon_2N_D+\epsilon_1N_A)V_{bi}]}$ \textbf{(30)}\\
$ I=I_0(1-V_a/V_{bi})[exp(V_a/V_t)-1]$ \textbf{(31)}\\
\end{tabular}


\subsection{3.3 - Schottky Junction}
\subsubsection{Schottky Junction at Equilibrium}
\begin{tabular}{@{}ll@{}}
$ f_{vx}=\sqrt{[m^*_e/(2\pi\Phi_t)]exp[-m^*_ev^2_x/(2\Phi_t)]}$ \textbf{(32)}\\
$ \Phi_t=kT$ \\
$ \langle v_x\rangle=\int_0^{\infty}[v_xf_{vx}]dv_x=\sqrt{\Phi_t/(2\pi m^*_e)}$ \textbf{(33)}\\
$ I=-qA_{cs}\int_0^\infty [v_x\partial n/\partial E]dE$ \textbf{(34)}\\
$ I_{sm}=A^*T^2A_{cs}exp[(-\Phi_{bi}-E_n)/\Phi_t]$ \textbf{(35)}\\
$ I_{ms}=-A^*T^2A_{cs}exp[-\Phi_{B}/\Phi_t]$ \textbf{(36)}\\
At equilib., no net current flowing $I_{sm}=-I_{ms}$ \\
give $\Phi_B=\Phi_{bi}+E_n$ \textbf{(37)}\\
\end{tabular}


\subsubsection{Schottky Junction under Bias}
\begin{tabular}{@{}ll@{}}
$ \Phi_{bi}=\Phi_B-E_n-qV_a$ \textbf{(38)}\\
Total fwd-bias current $ I=I_{sm}+I_{ms}$ \textbf{(39)}\\
$ =A^*T^2A_{cs}exp(-\Phi_{B}/\Phi_t)[exp(V_a/V_t)-1]$ \\
$ =I_0[exp(V_a/V_t)-1]$ \\
where $ I_0 (=A^*T^2A_{cs}exp(-\Phi_B/\Phi_t))$ is the saturation current \\
$ I_{tun}=I_fexp(\Phi_B/E_{\infty})$ \textbf{(40)}\\
$ E_{\infty}=qh/4\pi\sqrt{N_D/(\epsilon_sm^*_e)}$ \\
\end{tabular}


\subsubsection{Nonideal Schottky Junctions}
\begin{tabular}{@{}ll@{}}
Schottky barrier height $ \Phi_B=E_g-\Phi_0$ \textbf{(41)}\\
$ \Delta\Phi_B=q\sqrt{qE^{\prime}/(4\pi\epsilon_s)} $ \textbf{(42)}\\
$ I_p=(qD_pp_{n0}A_{cs}/L_p)[exp(V_a/V_t-1)]$ \textbf{(43)}\\
$ \gamma^*=I_p/I$ \textbf{(44)}\\
$ I=A^*T^2A_{cs}exp(-\Phi_B/\Phi_t)[exp(V_a/(\eta_IV_t))-1]$ \textbf{(45)}\\
\end{tabular}


\subsubsection{Capacitance Effect and Equivalent-Circuit Model of a Schottky Junction}
\begin{tabular}{@{}ll@{}}
$ C_{jun}=A_{cs}\sqrt{qN_D\epsilon_s}$ \textbf{(46)} \\
\end{tabular}


\subsubsection{Modification of the Barrier Height}
\begin{tabular}{@{}ll@{}}
$ \Phi^*_B=\Phi_B-q/\epsilon_s\sqrt{n_1a^{\prime}/4\pi}$ \textbf{(47)} \\
$ \Delta d=[a^{\prime}p_1-(W-a^{\prime})n_2]/p_1$ \textbf{(48)}\\
$ \Phi^*_B=\Phi_B+q^2p_1\Delta d^2/2\epsilon_s$ \textbf{(49)}\\
\end{tabular}


\subsection{3.4 - Metal-Semiconductor Contact}
$ R_c=\sqrt{R_{sh}r_c}/d_ccoth[L\sqrt{R_{sh}/r_c}]$ \textbf{(50)} \\
$ R_{sh}=r_{sheet}L_{sh}/d_c$ \textbf{(51)}\\
$ R=R_c+R_{sh}+R_{sp}$ \textbf{(52)}\\
$ r_c\approx exp[2V_{bn}\sqrt{\epsilon_sm^*_e/N_D}/(h^{\prime})]$ \textbf{(53)}\\


\subsection{3.5 - MIS Junction and Field-Effect Properties}
$ $ \textbf{(54)} \\
\subsubsection{Surface Inversion}
\begin{tabular}{@{}ll@{}}
$ $ \textbf{(55)} \\
$ $ \textbf{(56)} \\
$ $ \\
$ $ \\
$ $ \\
$ $ \\
$ $ \\
\end{tabular}

\section{Glossary of Variables}
\begin{tabular}{@{}ll@{}}
$\Phi_{bi}$ is the build-in potential\\
$V_{bi} (=\Phi_{bi}/q)$ is the build-in voltage in V \\
k($ 8.62$ x $10^{-5}ev/K$): Boltzmann constant \\
T: absolute temp. in Kelvin \\
$ n_i$ is the intrinsic carrier density of the semiconductor $/m^3$ \\
$ N_A$ is the acceptor density $/m^3$ \\
$ N_D$ is the donor density $/m^3$ \\
$ V_a$ is the applied voltage and is a fixed voltage in V \\
$ I_0$ is the saturation current in A \\
$ V_t (=kT/q)$ is the thermal voltage (0.026V unless said otherwise)\\
$ R_{dyn}$ is the dynamic resistance ($\Omega$) \\
$ I_r$ is the reverse current (or leakage current) \\
$ I_p$ is the hole diffusion current flowing into the n side in a $p^+n$ \\
q ($1.602$ x $10^{-19}$ C) is the electron charge in C \\
$ D_p$ is the hole diffusivity in $m^2/s$ \\
$ D_n$ is the electron diffusivity in $m^2/s$ \\
$ A_{cs}$ is the cross section of the pn junc in $m^2$ \\
$ d(p_n(x))/dx$ is the hole density grad. in the n-side in $/m^4$\\
$ p^{\prime}$ is the excess hole density at x=0 in $/m^3$ \\
$ p_{n0} (=n^2_i/N_D)$ is the equilib. hole density in the n-side in $/m^3$ \\
$ p_{p0} (=n^2_i/N_A)$ \\
$ L_p$ is the diffusion length of the holes in m \\
$ L_n$ is the diffusion length of the electrons in m \\
$ I_G$ thermal generation current \\
$ \tau_0$ is the generation lifetime of the carriers in s\\
$ \tau_p$ is the lifetime of the holes in s\\
$ W_j$ is the width of the junc. region in m\\
$ \eta_I$ is the ideality factor, takes into account the recomb. effect \\
$ \varepsilon_s$ is the semiconductor permittivity in F/m \\
$ C_{jun}$ is the junction capacitance in F \\
$ N_{eff}=N_AN_D/(N_A+N_D)$ \\
$ a$ is the gradient of the dopant density in $/m^4$ \\
$ V_0$ is the forward-bias voltage in V \\
$ D_{12}$ the transmission coefficient (prob. that tunnelling event will occur) \\
$ D_b$ is a constant \\
$ m^*_e$ is the effective mass of the electrons in kg \\
$ d_b$ is the barrier width in m \\
$ h^{\prime} (=1.05$ x $10^{-34}J\cdot s)$ is Planck's constant divided by $2\pi$\\
$ I_{tun}$ is the tunneling current in A \\
$ S_1 and S_2$ are the respective densities of states of E. bands in the two \\
sides of the PN junction in $ /m^3$ \\
$ f_t$ is a tunneling freq. parameter in $m^4/s\cdot eV$ \\
$ E_c$ is the E. at the conduction band edge in eV\\
$ \acute E$ is the upper E. limit for tunneling in eV \\
$ \acute{E}^2_{cr}$ is the critical field \\
$ V_{br}$ is the breakdown voltage in V \\
$ j_n$ is the electron current density in $A/m^2$ \\
$ j_p$ is the hole current density in $A/m^2$ \\
$ j_T (=j_n+j_p)$ is the total current density in $A/m^2$ \\
$ \alpha_i$ is the electron ionization rate in /m \\
$ \beta_i$ is the hole ionization rate in /m \\
L is the length of the ionization region in m \\
$ M_n$ is the electron multiplication factor \\
$ E_g$ is the E. gap of the semiconductor in eV \\
$ \hat{i}^2$ is the shot noise (generated when carriers cross a barrier) in $A^2$  \\
$ I_d$ is the diode current in A \\
$ I_0$ is the saturation current in A \\
\end{tabular}

\begin{tabular}{@{}ll@{}}
$ \delta f$ is the freq. range in Hz \\
$ \xi_1$ and $\xi_2$ are the respective electron affinities in the \\
two semiconductors \\
$ E_{g1}$ is the E. gap of the narrow-gap p-type semiconductor in eV \\
$ \Delta E_n$ is the E. difference between the conduction band edge and the Fermi \\
level in the wide-gap semiconductor in eV \\
$ \Delta E_p$ is the E. difference between the valence band edge and the Fermi \\
level in the narrow-gap semiconductor in eV \\
$ \epsilon_1$ and $\epsilon_2$ are the respective permittivities of the p-type and n-type \\
semiconductors in F/m \\
$ v_x$ is the velocity of the electrons in the +x direction in m/s \\
$ m^*_e$ is the effective mass of the electrons in kg \\
$ \langle v_x\rangle$ is the average velocity of electrons in m/s \\
$ f_{vx}$ is the velocity distribution of the electrons \\
$ \partial n/\partial E$ is the rate at which the electron density changes with E. band\\
in $/m^3\cdot eV$  \\
$ I_{sm}$ is the current of electrons moving from the semiconductor into the metal (pos. value)\\
$ A^* (=4\pi qm^*_ek^2/h^3=1.2$ x $10^6 A/m^2\cdot K^2$ is called the Richardson constant \\
$ E_n$ is the E. diff. between the semiconductor Fermi lvl and the conduction band edge in eV \\
$ \Phi_{B}$ is the barrier height in eV \\
$ E_{\infty}$ is a parameter dependent on the dopant density in J \\
$ I_f$ is a pre-exponential constant that depends on the field-emission process in A \\
$ \Phi_0$ is the location of the surface Fermi level in eV \\
$ E^{\prime}$ is the electric field in V/m \\
$ \Delta\Phi_B$ is the resulting lowering in the barrier height in eV \\
- usually quite small $(\approx 0.01eV)$  \\
$I_p$ is the minority carrier current in A \\
$ \gamma^*$ is the minority carrier injection ratio \\
I is the total current flowing across the Schottky junction in A \\
$a^{\prime}$ is the dopant layer thickness in m \\
$ n_1$ is the interface dopant density in $/m^3$ \\
$n_2$ is the substrate donor density in $/m^3$ \\
W is the depletion-layer width in m
$ \Delta d$ is the distance away from the interface in m \\
$ \Phi^*_B$ is the effective barrier height in m \\
$ p_1$ is the surface dopant density in $/m^3$ \\
$ p_2$ is the surface dopant density in $/m^3$ \\
$ R_c$ is the contact resistance in $\Omega$ \\
$ R_{sh}$ is the shunt resistance in $\Omega$ \\
$ r_c$ is the specific contact resistance in $\Omega\cdot m^2$ \\
$ d_c$ is the width of the contact in m \\
L is the contact length in m
$ r_{sheet}$ is the semiconductor sheet resistance in $\Omega$/square \\
$ L_{sh}$ is the length of the shunt path in m \\
$ m^*_e$ is the effective mass of the electrons in kg \\
$ V_{bn} (=\Phi_B/q)$ is in V \\
$ h^{\prime}$ is Planck's constant divided by $2\pi$ \\
$ N_D$ is the donor density in $/m^3$ \\
$ $ \\
$ $ \\
$ $ \\
$ $ \\
$ $ \\
$ $ \\
$ $ \\
$ $ \\
$ $ \\
$ $ \\
$ $ \\
$ $ \\
\end{tabular}


\end{multicols}
\end{document}
