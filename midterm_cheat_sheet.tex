\documentclass[10pt,landscape]{article}
\usepackage{multicol}
\usepackage{calc}
\usepackage{ifthen}
\usepackage[landscape]{geometry}
\usepackage{hyperref}
\usepackage{amsmath}
\usepackage{mathtools}

% To make this come out properly in landscape mode, do one of the following
% 1.
%  pdflatex latexsheet.tex
%
% 2.
%  latex latexsheet.tex
%  dvips -P pdf  -t landscape latexsheet.dvi
%  ps2pdf latexsheet.ps


% This sets page margins to .5 inch if using letter paper, and to 1cm
% if using A4 paper. (This probably isn't strictly necessary.)
% If using another size paper, use default 1cm margins.
\ifthenelse{\lengthtest { \paperwidth = 11in}}
	{ \geometry{top=.5in,left=.5in,right=.5in,bottom=.5in} }
	{\ifthenelse{ \lengthtest{ \paperwidth = 297mm}}
		{\geometry{top=1cm,left=1cm,right=1cm,bottom=1cm} }
		{\geometry{top=1cm,left=1cm,right=1cm,bottom=1cm} }
	}

% Turn off header and footer
\pagestyle{empty}

% Redefine section commands to use less space
\makeatletter
\renewcommand{\section}{\@startsection{section}{1}{0mm}%
                                {-1ex plus -.5ex minus -.2ex}%
                                {0.5ex plus .2ex}%x
                                {\normalfont\large\bfseries}}
\renewcommand{\subsection}{\@startsection{subsection}{2}{0mm}%
                                {-1explus -.5ex minus -.2ex}%
                                {0.5ex plus .2ex}%
                                {\normalfont\normalsize\bfseries}}
\renewcommand{\subsubsection}{\@startsection{subsubsection}{3}{0mm}%
                                {-1ex plus -.5ex minus -.2ex}%
                                {1ex plus .2ex}%
                                {\normalfont\small\bfseries}}
\makeatother

% Define BibTeX command
\def\BibTeX{{\rm B\kern-.05em{\sc i\kern-.025em b}\kern-.08em
    T\kern-.1667em\lower.7ex\hbox{E}\kern-.125emX}}

% Don't print section numbers
\setcounter{secnumdepth}{0}


\setlength{\parindent}{0pt}
\setlength{\parskip}{0pt plus 0.5ex}


% -----------------------------------------------------------------------

\begin{document}

\raggedright
\footnotesize
\begin{multicols}{3}


% multicol parameters
% These lengths are set only within the two main columns
%\setlength{\columnseprule}{0.25pt}
\setlength{\premulticols}{1pt}
\setlength{\postmulticols}{1pt}
\setlength{\multicolsep}{1pt}
\setlength{\columnsep}{2pt}

\begin{center}
     \Large{\textbf{ELEC412 Cheat Sheet}} \\
\end{center}


\section{Chapter 3: PN Junctions}
\subsection{3.2 - Concept of PN Junction}
\subsubsection{Energy Band Diagram for a PN Junction}
\begin{tabular}{@{}ll@{}}
Built in potential: $\Phi_{bi}=kTln(\frac{N_AN_D}{n^2_i})$ \textbf{(1)}\\
\end{tabular}


\subsubsection{I-V Characteristics of a PN Junction}
\begin{tabular}{@{}ll@{}}
$I=I_0exp(V_a/V_t)$ \textbf{(2)}\\
Dynamic resistance: $R_{dyn}=\frac{d(V_a)}{dI}=\frac{V_t}{I}$ \textbf{(3)}\\
IV characteristics of pn junc. w/ rev. current: \\
$I=I_0exp(V_a/V_t)-I_r$ \textbf{(4)}\\
If $I_r=I_0$ then $I=I_0[exp(V_a/V_t)-1]$ - ideal diode eqn.\textbf{(5)}\\
Hole diffusion current flowing into the n-side in a $p^+n$ junc.: \\
$I_p(x)=qD_p[d(p_n(x))/dx]A_{cs}$ \textbf{(6)}\\
$ p_n(x)=p^{\prime}exp(-x/L_p)+p_{n0}$ \textbf{(7)}\\
$ d(p_n(x))/dx=-(p_n(x)-p_{n0})/L_p$ \textbf{(8)}\\
$ I_p(x)=-qD_p(p_n(x)-p_{n0})A_{cs}/L_p$ \textbf{(9)}\\
$ p^{\prime}=p_{n0}[exp(qV_a/kT)-1]$ \textbf{(10)}\\
$ I_p=[qD_pn^2_iA_{cs}/(N_DL_p)][exp(qV_a/kT)-1]$ \textbf{(11)}\\
$ I=I_p+I_n=I_0[exp(V_a/V_t)]-1$ \textbf{(12)}\\
where $ I_0=qD_pn^2_iA_{cs}/(N_DL_p)+qD_nn^2_iA_{cs}/(N_AL_p)$ \\
$ I_G=qn_iW_jA_{cs}/(2\tau_0)$ \textbf{(13)}\\
$ I=I_0exp[(V_a/\eta_IV_t)-1]-I_G$ \textbf{(14)}\\
\end{tabular}


\subsubsection{Space-Charge and Charge Storage Effects}
\begin{tabular}{@{}ll@{}}
$ C_{jun}=\varepsilon_sA_{cs}/W_j$ \textbf{(15)}\\
$ W_j=\sqrt{2\varepsilon_s(V_{bi}-V_a)/(qN_{eff})}$ \textbf{(16)}\\
$ W_j=[12\varepsilon_s(V_{bi}-V_a)/(qa)]^{1/3}$ \textbf{(17)}\\
$ C_{diff}=[qA_{cs}(L_pp_{n0}+L_np_{p0})/2V_t]exp(V_0/V_t)$ \textbf{(18)}\\
$ C_{diff}\approx I_0\tau_p/(2V_t)$ \textbf{(19)}\\
\end{tabular}


\subsubsection{Tunnel PN Junctions}
\begin{tabular}{@{}ll@{}}
$ D_{12}=D_bexp(-2d_b\sqrt{2m^*_e(\Phi_B-E})/h^{\prime})$ \textbf{(20)}\\
$ I_{tun}=qA_{cs}\int_{E_c}^{\acute E}[f_tS_1D_{12}S_2]dE$ \textbf{(21)}\\
\end{tabular}


\subsubsection{Junction Breakdown}
\begin{tabular}{@{}ll@{}}
$ V_{br}=\epsilon_s \acute{E}^2_{cr}/(2qN_{eff})$ \textbf{(22)}\\
$ d(j_n)/dx-(\alpha_i-\beta_i)j_n=-(\alpha_i-\beta_i)j_T+\alpha_ij_T$ \textbf{(23)}\\
If $ \alpha_i=\beta_i$ then $ M_n=j_n(L)/j_n(0)=1/[1-\int_0^L(\alpha_i)dx]$ \textbf{(24)}\\
$ V_{br}\approx 4E_g/q$ \textbf{(25)}\\
\end{tabular}


\subsubsection{Noise in PN Junctions}
\begin{tabular}{@{}ll@{}}
$ \hat{i}^2=2q(I_d+2I_0)\delta f$ \textbf{(26)}\\
\end{tabular}


\subsubsection{Heterojunctions}
\begin{tabular}{@{}ll@{}}
$ \Delta E_c = \xi_1-\xi_2$ \textbf{(27)}\\
$ \phi_{bi}=E_{g1}-\Delta E_n-\Delta E_p+\Delta E_c$ \textbf{(28)}\\
$ W_j=W_n+W+p$ \textbf{(29)}\\
$ W_n=\sqrt{2\epsilon_1\epsilon_2N_AV_{bi}/[qN_D(\epsilon_2N_D+\epsilon_1N_A)]}$ \\
$ W_p=\sqrt{2\epsilon_1\epsilon_2N_DV_{bi}/[qN_A(\epsilon_2N_D+\epsilon_1N_A)]}$ \\
$ C_{jun}=\sqrt{\epsilon_1\epsilon_2N_DN_A/[2(\epsilon_2N_D+\epsilon_1N_A)V_{bi}]}$ \textbf{(30)}\\
$ I=I_0(1-V_a/V_{bi})[exp(V_a/V_t)-1]$ \textbf{(31)}\\
\end{tabular}


\subsection{3.3 - Schottky Junction}
\subsubsection{Schottky Junction at Equilibrium}
\begin{tabular}{@{}ll@{}}
$ f_{vx}=\sqrt{[m^*_e/(2\pi\Phi_t)]exp[-m^*_ev^2_x/(2\Phi_t)]}$ \textbf{(32)}\\
$ \Phi_t=kT$ \\
$ \langle v_x\rangle=\int_0^{\infty}[v_xf_{vx}]dv_x=\sqrt{\Phi_t/(2\pi m^*_e)}$ \textbf{(33)}\\
$ I=-qA_{cs}\int_0^\infty [v_x\partial n/\partial E]dE$ \textbf{(34)}\\
$ I_{sm}=A^*T^2A_{cs}exp[(-\Phi_{bi}-E_n)/\Phi_t]$ \textbf{(35)}\\
$ I_{ms}=-A^*T^2A_{cs}exp[-\Phi_{B}/\Phi_t]$ \textbf{(36)}\\
At equilib., no net current flowing $I_{sm}=-I_{ms}$ \\
give $\Phi_B=\Phi_{bi}+E_n$ \textbf{(37)}\\
\end{tabular}


\subsubsection{Schottky Junction under Bias}
\begin{tabular}{@{}ll@{}}
$ \Phi_{bi}=\Phi_B-E_n-qV_a$ \textbf{(38)}\\
Total fwd-bias current $ I=I_{sm}+I_{ms}$ \textbf{(39)}\\
$ =A^*T^2A_{cs}exp(-\Phi_{B}/\Phi_t)[exp(V_a/V_t)-1]$ \\
$ =I_0[exp(V_a/V_t)-1]$ \\
where $ I_0 (=A^*T^2A_{cs}exp(-\Phi_B/\Phi_t))$ is the saturation current \\
$ I_{tun}=I_fexp(\Phi_B/E_{\infty})$ \textbf{(40)}\\
$ E_{\infty}=qh/4\pi\sqrt{N_D/(\epsilon_sm^*_e)}$ \\
\end{tabular}


\subsubsection{Nonideal Schottky Junctions}
\begin{tabular}{@{}ll@{}}
Schottky barrier height $ \Phi_B=E_g-\Phi_0$ \textbf{(41)}\\
$ \Delta\Phi_B=q\sqrt{qE^{\prime}/(4\pi\epsilon_s)} $ \textbf{(42)}\\
$ I_p=(qD_pp_{n0}A_{cs}/L_p)[exp(V_a/V_t-1)]$ \textbf{(43)}\\
$ \gamma^*=I_p/I$ \textbf{(44)}\\
$ I=A^*T^2A_{cs}exp(-\Phi_B/\Phi_t)[exp(V_a/(\eta_IV_t))-1]$ \textbf{(45)}\\
\end{tabular}


\subsubsection{Capacitance Effect and Equivalent-Circuit Model of a Schottky Junction}
\begin{tabular}{@{}ll@{}}
$ C_{jun}=A_{cs}\sqrt{qN_D\epsilon_s}$ \textbf{(46)} \\
\end{tabular}


\subsubsection{Modification of the Barrier Height}
\begin{tabular}{@{}ll@{}}
$ \Phi^*_B=\Phi_B-q/\epsilon_s\sqrt{n_1a^{\prime}/4\pi}$ \textbf{(47)} \\
$ \Delta d=[a^{\prime}p_1-(W-a^{\prime})n_2]/p_1$ \textbf{(48)}\\
$ \Phi^*_B=\Phi_B+q^2p_1\Delta d^2/2\epsilon_s$ \textbf{(49)}\\
\end{tabular}


\subsection{3.4 - Metal-Semiconductor Contact}
$ R_c=\sqrt{R_{sh}r_c}/d_ccoth[L\sqrt{R_{sh}/r_c}]$ \textbf{(50)} \\
$ R_{sh}=r_{sheet}L_{sh}/d_c$ \textbf{(51)}\\
$ R=R_c+R_{sh}+R_{sp}$ \textbf{(52)}\\
$ r_c\approx exp[2V_{bn}\sqrt{\epsilon_sm^*_e/N_D}/(h^{\prime})]$ \textbf{(53)}\\


\subsection{3.5 - MIS Junction and Field-Effect Properties}
$ V_{FB}=[\Phi_m-\Phi_s-(E_c-E_f)]/q$ \textbf{(54)} \\
\subsubsection{Surface Inversion}
\begin{tabular}{@{}ll@{}}
$ V_m=V_s+\sqrt{2\epsilon_sqN_AV_s}/C_i$ \textbf{(55)} \\
$ n_s=n_{p0}exp(V_s/V_t)$ \textbf{(56)} \\
$ p_s=p_{p0}exp(-V_s/V_t)$ \\
$ V_t=kT/q$ \\
$ \acute E_s=\sqrt{2}(V_t/L_{D_p})\cdots$ \textbf{(57)} \\
$\cdot\sqrt{[exp(-V_s/V_t)+V_s/V_t-1]n_{p0}/p_{p0}[exp(V_s/V_t)-V_s/V_t-1]}$ \\
$ V_a=2\phi_b-\sqrt{4qN_A\phi_b\epsilon_s}/C_i$ \textbf{(58)} \\
$ V^*_T=V_T+V_{FB}+V_{sc}+V_{sub}$ \textbf{(59)} \\
\end{tabular}


\subsubsection{Capacitance Effect in a MIS Junction}
\begin{tabular}{@{}ll@{}}
$ C_{mis}=C_iC_{depl}/(C_i+C_{depl})$ \textbf{(60)} \\
\end{tabular}


\subsection{3.7 - Structure and Operations of Transistors}
\subsubsection{Field-Effect Transistors (FETs)}
\begin{tabular}{@{}ll@{}}
$ Q_s=-C_i[V_g-2\phi_b-V_{FB}-V_c(x)]$ \textbf{(61)} \\
$ n_{ss}(x)=Q_s-\sqrt{2\epsilon_sqN_A(V_c(x)+2\phi_b)}$ \textbf{(62)} \\
$ I_{ds}=qn_{ss}W_c\mu_ndV_c(x)/dx$ \textbf{(63)} \\
$ I_{ds}=(\mu_nW_c/L)C_i \{ (V_g-V_{FB}-2\phi_b-V_{ds}/2)-(2/3)\cdots$ \textbf{(64)} \\
$ \cdot[\sqrt{2\epsilon_sqN_A}/C_i][(V_{ds}+2\phi_b)^{3/2}-(2\phi_b)^{3/2}] \}$ \\
$ V_{ds}=V_g-V_T$ \textbf{(65)} \\
$ V_T=V_{FB}+2\phi_b+V_{sc}+\sqrt{2\epsilon_sqN_A(2\phi_b-V_{sub})}/C_i$ \textbf{(66)} \\
$ I_{ds}\approx \beta_0(V_g-V_T)V_{ds}$ \textbf{(67)} \\
$ \beta_0=\mu_nC_iW_c/L$ \\
$ I_{D_{sat}}\approx \beta_0(V_g-V_T)^2/2$ \textbf{(68)} \\
$ g_m=\beta_0(V_g-V_T)$ \textbf{(69)} \\
$ I_{ds}=I_{D_{sat}}tanh(g_{ds}V_{ds}/I_{D_{sat}})[1+\gamma^{\prime}V_{ds}]$ \textbf{(70)} \\
$ \Delta I_{ds}=g_m\Delta V_g=\beta_0(V_g-V_T)\Delta V_g$ \textbf{(71)} \\
$ G_v=\beta_0(V_g-V_T)R_L$ \textbf{(72)} \\
$ \Delta V=I_{ds}\Delta x/[q\mu_nN_DW_c(t_c-a_d(x))]$ \textbf{(73)} \\
$ a_d(x)=2\epsilon_s(V(x)+V_{bi}-V_g)/(qN_D)$ \\
$ I_{ds}=g_0[V_{ds}-2((V_{ds}+V_{bi}-V_g)^{3/2}-(V_{bi}-V_g)^{3/2})/(3\sqrt{V_{po}})]$ \textbf{(74)} \\
$ g_m=g_0(\sqrt{V_{ds}+V_{bi}-V_g}-\sqrt{V_{bi}-V_g}/\sqrt{V_{po}})$ \textbf{(75)} \\
$ V_{D_{sat}}=V_{po}-V_{bi}+V_g$ \\
$ I_{D_{sat}}=g_0(V_{po}/3+2(V_{bi}-V_g)^{3/2}/(3\sqrt{V_{po}})-V_{bi}+V_g)$ \textbf{(76)} \\
$ g_m=g_0[1-\sqrt{(V_{bi}-V_g)/V_{po}}]$ \\
$ I_{ds}=\beta^{\prime}(V_g-V_T)^2(1+\gamma^{\prime}V_{ds})tanh(g_0V_{ds}/I_{D_{sat}})/[1+b(V_g-V_T)]$ \textbf{(77)} \\
$ \beta^{\prime}=\mu_n\epsilon_sW/(t_cL)$ \\
$ V_T=V_{bi}-V_{po}$ \\
$ I_{gate}=I_{gs}+I_{gd}$ \textbf{(78)} \\
$ I_{gs}=I_{go}exp\{[V_{gs}-I_gR_g-(I_{ds}+I_{gs})R_s](\eta_gV_{th})\}$ \\
$ I_{gd}=I_{go}exp\{[V_{gd}-I_gR_g-(I_{ds}+I_{gd})R_s](\eta_gV_{th})\}$ \\
\end{tabular}


\subsubsection{Bipolar Junction Transistor (BJT)}
\begin{tabular}{@{}ll@{}}
$ D_{pb}d^2(p_b(x)-p_{n0})/dx^2-(p_b(x)-p_{n0})/\tau_p=0$ \textbf{(79)} \\
$ p_b(x)=p_{n0}+A_{b1}exp(x/L_{pb})+A_{b2}exp(-x/L_{pb})$ \textbf{(80)} \\
$ p_b(x)=p_{n0}[exp(V_a/V_t-1)]sinh[(W^{\prime}-x/L_{pb})]/sinh[W^{\prime}/L_{pb}]$ \textbf{(81)} \\
$ \cdots +p_{n0}[1-sinh(x/L_{pb})/sinh(W^{\prime}/L_{pb})]$ \\
$ p_b(x)\approx p_{n0}[exp(V_a/V_t-1)][1-x/W^{\prime}]$ \textbf{(82)} \\
$ I_{pb}=-qD_{pb}d(p_b(x))/dxA_{cs}=qD_{pb}p_{n0}[exp(V_a/V_t-1)]A_{cs}/W^{\prime}$ \textbf{(83)} \\
$ n_e(x)=n_{p0}+n_{p0}[exp(V_a/V_t)-1]exp[(x+x_e)/L_{ne}]$ \textbf{(84)} \\
$ n_c(x)=n_{p0}+n_{p0}[exp(-(x-x_c)/L_{nc})]$ \\
$ I_{ne}=qD_{ne}n_{p0}A_{cs}[exp(V_a/V_t-1)]/L_{ne}$ \textbf{(85)} \\
$ I_{nc}=qD_{nc}n_{p0}A_{cs}/L_{nc}$ \\
During device operation:
$ I_e\approx I_{ne}$ \\
$ I_c=I_{nc}$ \textbf{(86)} \\
$ I_b=I_{pb}$ \\
$ \beta_g\approx D_{pb}N_{Ae}x_e/(D_{ne}N_{Db}W^{\prime})$ \textbf{(87)} \\
$ Q_G=\int_0^{W^{\prime}}[N_{Db}(x)]dx\approx W^{\prime}N_{Db}$ \textbf{(88)} \\
$ n_b(0)=p_b(0)\approx n_iexp(V_a/2V_t)$ \textbf{(89)} \\
$ I_e\approx I_c\approx (qn_iD_{pb}A_{cs}/W^{\prime})exp(V_a/2V_t)$ \textbf{(90)} \\
$ I_b\approx qD_{ne}n^2_iA_{cs}/(x_eN_{Ae})exp(V_a/V_t)$ \\
\end{tabular}


\subsection{3.8 - Nonideal Effects and Other Performance Parameters}
\subsubsection{MOSFETs}
\begin{tabular}{@{}ll@{}}
$ \mu^*=\mu_0/(1+\mu_0 \acute{E}/v_s)$ \textbf{(91)} \\
$ I_{ds}=qn_{ss}v_n(\acute{E})W_c$ \textbf{(92)} \\
$ V_c(x)=V_{gt}-\sqrt{V^2_{gt}-2I_{ds}x/\beta_0L}$ \textbf{(93)} \\
$ V_{gt}=V_g-V_T$ \\
$ \beta_0=\mu^*C_iW_c/L$ \\
$ I_{D_{sat}}=\beta_{sl}V^2_{sl}[\sqrt{(1+(V_{gt}/V_{sl}))^2}-\beta_{sl}V^2_{sl}]$ \textbf{(94)} \\
$ V_{sl}=\acute{E}_pL$ \\
$ I_{sub}=-qA_{cs}D_n\Delta n/\Delta x$ \textbf{(95)} \\
$ =\epsilon_s\mu_n(W_c/L)V^2_t(n_i/N_A)^2\sqrt{V_t/V_s}exp(V_s/V_t)\cdots$ \\
$\cdot[1-exp(-V_{ds}/V_t)]/(\sqrt{2}L_{Dp})$ \\
In saturation, these capacitances are given by:\\
$ C_{gs}=C_{gsf}+2C_iW_cL/3$ \\
$ C_{gd}=C_{gdf}$ \\
$ C_{gb}=0$ \\
$ C_{bs}=C_{js}(1+\frac{2}{3}C_{gb}/C_{js}W_cL)/(1+V_{sub-s}/\phi_{bi})^{mB}$ \\
$ C_{bd}=C_{jd}/(1+V_{sub-d}/\phi_{bi})^{mB}$ \textbf{(96)} \\
In the linear regime, the same parameters are given by:\\
$ C_{gs}=C_{gsf}+\frac{2}{3}C_iW_cLV_{D_{sat}}(3V_{D_{sat}}-2V_{ds})/(2V_{D_{sat}}-V_{ds})^2$ \\
$ C_{gs}=C_{gdf}+\frac{2}{3}C_iW_cL(V_{D_{sat}}-V_{ds})(3V_{D_{sat}}-V_{ds})\cdots$ \\
$/(2V_{D_{sat}}-V_{ds})^2$ \\
$ C_{gb}=0$ \textbf{(97)} \\
$ C_{bs}=C_{js}[1+\frac{2}{3}C_{gb}/C_{js}W_cLV_{D_{sat}}(3V_{D_{sat}}-2V_{ds})\cdots$ \\
$/(2V_{D_{sat}}-V_{ds})^2]/(1+V_{sub-s}/\phi_{bi})^{mB}$ \\
$ C_{bd}=C_{jd}[1+\frac{2}{3}C_{gb}/C_{js}W_cL(V_{D_{sat}}-V_{ds})(3V_{D_{sat}}-V_{ds})\cdots$ \\
$/(2V_{D_{sat}}-V_{ds})^2]/(1+V_{sub-s}/\phi_{bi})^{mB}$ \\
$ f_T=g_m/[2\pi (C_g+C_p)]$ \textbf{(98)} \\
$ \Delta V_{TE}=\Delta V_{bi}+q\Delta n_A/C_i$ \textbf{(99)} \\
$ \Delta V_{TD}=\Delta V_{bi}-\Phi_x/q-q\Delta n_D/C_i$ \\
$ \hat{i}^2_{th}=4kTg_{do}$ \textbf{(100)} \\
$ C^{\prime}=C_i/k^*$ \\
$ I^{\prime}_{D_{sat}}=I_{D_{sat}}/k^*$ \textbf{(101)} \\
$ f^{\prime}_T=k^*f_t$ \\
$ P^{\prime}_{ac}=P_{ac}/(k^*)^2$ \\
\end{tabular}


\subsubsection{MESFETs}
\begin{tabular}{@{}ll@{}}
$ I_{D_{sat}}=\beta^{\prime\prime}(V_G-V_T)^2$ \textbf{(102)} \\
$ \beta^{\prime\prime}=2\epsilon_s\mu_nv_sW_c/[t_c(\mu_nV_{po}+3v_sL)]]$ \\
$ V_T=V_{bi}-V_{po}$ \\
$ C_{gs}=C_{go}/\sqrt{1-V_{gs}/V_{bi}}+\pi\epsilon_sW_c/2$ \textbf{(103)} \\
$ C_{gd}=C_{go}/\sqrt{1-V_{gd}/V_{bi}}+\pi\epsilon_sW_c/2$ \\
$ L/t_c<3$ and $N_DL\approx 1.6$ x $10^{23} \mu m/m^3$ \textbf{(104)} \\
$ $ \\
$ $ \\
$ $ \\
\end{tabular}


\subsubsection{BJTs}
\begin{tabular}{@{}ll@{}}
$ \Delta E_g=3q^2\sqrt{qN_{De}/\epsilon_s\cdot V_t}/16\pi\epsilon_s$ \textbf{(105)} \\
$ N_{Ab}\approx N_{Ao}exp[-(x-x_e)/\lambda_b]$ \textbf{(106)} \\
$ \beta^*\approx\beta_gW^{\prime}/2\lambda_b$ \textbf{(107)} \\
$ \Delta Q_{bs}=(I_{b1}-I_{b2})\tau_{sr}exp(-t/\tau_{sr})+(I_{b2}-I_{ba})\tau_{sr}$ \textbf{(108)} \\
$ \tau_s=\tau_{sr}ln[(I_{b1}-I_{b2})/(I_{ba}-I_{b2})]$ \textbf{(109)} \\
$ f_{\beta}=(C_er_e)/2\pi$ \textbf{(110)} \\
$ f_{\beta}=g_{b^{\prime}e}/2\pi (C_{b^{\prime}e}+C_{b^{\prime}c})$ \textbf{(111)} \\
$ f_{tr}=<v>/W^{\prime}$ \textbf{(112)} \\
$ BV_{cb}=\epsilon_s\acute{E}^2_{cr}/(2qN_{Dc})$ \textbf{(113)} \\
$ BV_{ce}=BV_{cb}(1-\alpha_g)^{1/mb}$ \textbf{(114)} \\
$ V_{pt}=qN_{Ab}W^{\prime 2}/(2\epsilon_sN_{Dc})$ \textbf{(115)} \\
$ \hat{i}^2_1=2q(I_e+2I_b)\partial f$ \textbf{(116)} \\
$ \hat{i}^2_2=2qI_c\partial f$ \\
\end{tabular}


\subsection{3.9 - New Transistor Structures}
\subsubsection{Heterojunction Bipolar Transistor (HBT)}
\begin{tabular}{@{}ll@{}}
$ \beta_{max}=v_{nb}N_{De}x_e/(v_{pe}N_{Ab}W^{\prime})exp(\Delta E_v/(qV_t))$ \textbf{(117)} \\
$ \tau_d=2.5R_BC_{b^{\prime}c}+R_B\tau_B/R_L+(3C_{b^{\prime}c}+C_L)R_L$ \textbf{(118)} \\
$ f_T=1/[2\pi\tau_{tr}(1+C_L/C_g)]$ \textbf{(119)} \\
$ V_T=\frac{(\Phi_B-\Delta E_c)}{q}-qN_Dd^2_{dd}/(2\epsilon_1)$ \textbf{(120)} \\
$ V_T=\frac{(\Phi_B-\Delta E_c)}{q}-\frac{qn_sd_d}{\epsilon_1}$ \textbf{(121)} \\
$ I_{ds}=q\mu_nn_{xs}d(V_a)/dx$ \textbf{(122)} \\
$ n_{xs}=n_s-\epsilon_1V_a(x)/(qd_{eff})$ \\
$ I_{ds}=\mu\epsilon_1(W_c/L)[(V_g-V_T)V_{ds}/d_{eff}-V^2_{ds}/(2d_{eff})]$ \textbf{(123)} \\
$ V_{D_{sat}}=(qn_s/\epsilon_1)(1+a^{\prime}-\sqrt{1+a^{\prime 2}})$ \textbf{(124)} \\
$ I_{D_{sat}}=qn_s\mu_n\acute{E}_sW_c(\sqrt{1+a^{\prime 2}}-a^{\prime})$ \\
$ \acute{E}_s=v_s/\mu_n$ \\
$ a^{\prime}=\epsilon_1v_sL/(qn_s\mu_nd_{eff})$ \\
\end{tabular}


\subsubsection{Amorphous Silicon Thin-Film Transistor (TFT)}
\begin{tabular}{@{}ll@{}}
$ Q_D=qg_{vd}E_{donor}exp[(E_v-E_F)/E_{donor}]$ \textbf{(125)} \\
$ Q_A=qg_{cd}E_{acceptor}exp[(E_F-E_c)/E_{acceptor}]$ \\
$ E_c-E_{F0}=E_{acceptor}/(E_{acceptor}-E_{donor})\cdots$ \textbf{(126)} \\
$\cdot [E_g-E_{donor}ln(g_{vd}E_{donor})/(g_{cd}E_{acceptor})]$ \\
$ I_{ds}=q\mu n_sd(V_a)/dx W_c$ \textbf{(127)} \\
$ n_{ind}=n_{inds}-\epsilon_iV_a/(qd_i)$ \textbf{(128)} \\
$ I_{ds}=q\mu^{**}W_c/Ln_{inds}V_{ds}$ \textbf{(129)} \\
$ V_{D_{sat}}=qn_{inds}d_i/\epsilon_i$ \textbf{(130)} \\
\end{tabular}


\section{Glossary of Variables}
\begin{tabular}{@{}ll@{}}
$\Phi_{bi}$ is the build-in potential\\
$V_{bi} (=\Phi_{bi}/q)$ is the build-in voltage in V \\
k($ 8.62$ x $10^{-5}ev/K$): Boltzmann constant \\
T: absolute temp. in Kelvin \\
$ n_i$ is the intrinsic carrier density of the semiconductor $/m^3$ \\
$ N_A$ is the acceptor density $/m^3$ \\
$ N_D$ is the donor density $/m^3$ \\
$ V_a$ is the applied voltage and is a fixed voltage in V \\
$ I_0$ is the saturation current in A \\
$ V_t (=kT/q)$ is the thermal voltage (0.026V unless said otherwise)\\
$ R_{dyn}$ is the dynamic resistance ($\Omega$) \\
$ I_r$ is the reverse current (or leakage current) \\
$ I_p$ is the hole diffusion current flowing into the n side in a $p^+n$ \\
q ($1.602$ x $10^{-19}$ C) is the electron charge in C \\
$ D_p$ is the hole diffusivity in $m^2/s$ \\
$ D_n$ is the electron diffusivity in $m^2/s$ \\
$ A_{cs}$ is the cross section of the pn junc in $m^2$ \\
$ d(p_n(x))/dx$ is the hole density grad. in the n-side in $/m^4$\\
$ p^{\prime}$ is the excess hole density at x=0 in $/m^3$ \\
$ p_{n0} (=n^2_i/N_D)$ is the equilib. hole density in the n-side in $/m^3$ \\
$ p_{p0} (=n^2_i/N_A)$ \\
$ L_p$ is the diffusion length of the holes in m \\
$ L_n$ is the diffusion length of the electrons in m \\
$ I_G$ thermal generation current \\
$ \tau_0$ is the generation lifetime of the carriers in s\\
$ \tau_p$ is the lifetime of the holes in s\\
$ W_j$ is the width of the junc. region in m\\
$ \eta_I$ is the ideality factor, takes into account the recomb. effect \\
$ \varepsilon_s$ is the semiconductor permittivity in F/m \\
$ C_{jun}$ is the junction capacitance in F \\
$ N_{eff}=N_AN_D/(N_A+N_D)$ \\
$ a$ is the gradient of the dopant density in $/m^4$ \\
$ V_0$ is the forward-bias voltage in V \\
$ D_{12}$ the transmission coefficient (prob. that tunnelling event will occur) \\
$ D_b$ is a constant \\
$ m^*_e$ is the effective mass of the electrons in kg \\
$ d_b$ is the barrier width in m \\
$ h^{\prime} (=1.05$ x $10^{-34}J\cdot s)$ is Planck's constant divided by $2\pi$\\
$ I_{tun}$ is the tunneling current in A \\
$ S_1 and S_2$ are the respective densities of states of E. bands in the two \\
sides of the PN junction in $ /m^3$ \\
$ f_t$ is a tunneling freq. parameter in $m^4/s\cdot eV$ \\
$ E_c$ is the E. at the conduction band edge in eV\\
$ \acute E$ is the upper E. limit for tunneling in eV \\
$ \acute{E}^2_{cr}$ is the critical field \\
$ V_{br}$ is the breakdown voltage in V \\
$ j_n$ is the electron current density in $A/m^2$ \\
$ j_p$ is the hole current density in $A/m^2$ \\
$ j_T (=j_n+j_p)$ is the total current density in $A/m^2$ \\
$ \alpha_i$ is the electron ionization rate in /m \\
$ \beta_i$ is the hole ionization rate in /m \\
L is the length of the ionization region in m \\
$ M_n$ is the electron multiplication factor \\
$ E_g$ is the E. gap of the semiconductor in eV \\
$ \hat{i}^2$ is the shot noise (generated when carriers cross a barrier) in $A^2$  \\
$ I_d$ is the diode current in A \\
$ I_0$ is the saturation current in A \\
\end{tabular}

\begin{tabular}{@{}ll@{}}
$ \delta f$ is the freq. range in Hz \\
$ \xi_1$ and $\xi_2$ are the respective electron affinities in the \\
two semiconductors \\
$ E_{g1}$ is the E. gap of the narrow-gap p-type semiconductor in eV \\
$ \Delta E_n$ is the E. difference between the conduction band edge \\
and the Fermi level in the wide-gap semiconductor in eV \\
$ \Delta E_p$ is the E. difference between the valence band edge and \\
the Fermi level in the narrow-gap semiconductor in eV \\
$ \epsilon_1$ and $\epsilon_2$ are the respective permittivities of the \\
p-type and n-type semiconductors in F/m \\
$ v_x$ is the velocity of the electrons in the +x direction in m/s \\
$ m^*_e$ is the effective mass of the electrons in kg \\
$ \langle v_x\rangle$ is the average velocity of electrons in m/s \\
$ f_{vx}$ is the velocity distribution of the electrons \\
$ \partial n/\partial E$ is the rate at which the electron density changes \\
with E. band in $/m^3\cdot eV$  \\
$ I_{sm}$ is the current of electrons moving from the semiconductor \\
into the metal (pos. value)\\
$ A^* (=4\pi qm^*_ek^2/h^3=1.2$ x $10^6 A/m^2\cdot K^2$ is called the \\
Richardson constant \\
$ E_n$ is the E. diff. between the semiconductor Fermi lvl and the \\
conduction band edge in eV \\
$ \Phi_{B}$ is the barrier height in eV \\
$ E_{\infty}$ is a parameter dependent on the dopant density in J \\
$ I_f$ is a pre-exponential constant that depends on the \\
field-emission process in A \\
$ \Phi_0$ is the location of the surface Fermi level in eV \\
$ E^{\prime}$ is the electric field in V/m \\
$ \Delta\Phi_B$ is the resulting lowering in the barrier height in eV \\
- usually quite small $(\approx 0.01eV)$  \\
$I_p$ is the minority carrier current in A \\
$ \gamma^*$ is the minority carrier injection ratio \\
I is the total current flowing across the Schottky junction in A \\
$a^{\prime}$ is the dopant layer thickness in m \\
$ n_1$ is the interface dopant density in $/m^3$ \\
$n_2$ is the substrate donor density in $/m^3$ \\
W is the depletion-layer width in m \\
$ \Delta d$ is the distance away from the interface in m \\
$ \Phi^*_B$ is the effective barrier height in m \\
$ p_1$ is the surface dopant density in $/m^3$ \\
$ p_2$ is the surface dopant density in $/m^3$ \\
$ R_c$ is the contact resistance in $\Omega$ \\
$ R_{sh}$ is the shunt resistance in $\Omega$ \\
$ r_c$ is the specific contact resistance in $\Omega\cdot m^2$ \\
$ d_c$ is the width of the contact in m \\
L is the contact length in m \\
$ r_{sheet}$ is the semiconductor sheet resistance in $\Omega$/square \\
$ L_{sh}$ is the length of the shunt path in m \\
$ m^*_e$ is the effective mass of the electrons in kg \\
$ V_{bn} (=\Phi_B/q)$ is in V \\
$ h^{\prime}$ is Planck's constant divided by $2\pi$ \\
$ N_D$ is the donor density in $/m^3$ \\
$V_{FB}$ is the flat-band voltage in V (can be pos. and neg.) \\
$ \Phi_m$ is the metal work function in eV \\
$ \Phi_s$ is the semiconductor work function in eV \\
$ E_c-E_F$ is the E. diff. between the conduction band edge \\
and the Fermi lvl in the semiconductor in eV \\
$ V_s$ is the potential at the semiconductor surface in V \\
$ C_i$ is the oxide capacitance per unit are in $F/m^2$ \\
\end{tabular}

\begin{tabular}{@{}ll@{}}
$ n_{p0}$ is the equilib. electron density in the \\
p-type semiconductor in $/m^3$ \\
$ p_{p0}$ is the equilib. hole density in the p-type \\
semiconductor in $/m^3$ \\
$ p_s$ is the surface charge density for the electrons in $/m^3$ \\
$ n_s$ is the surface charge density for the holes in $/m^3$ \\
$ \acute{E_s}$ is the electric field at the semiconductor's surface in V/m \\
$ L_{D_p}$ is the hole-diffusion length in m \\
$ V_T$ is the threshold voltage in V \\
$ V^*_T$ is the modified threshold voltage in V \\
$ V_{sc}$ is the potential drop due to oxide charges in the insulator in V \\
$ V_{FB}$ is the flat-band voltage in V \\
$ V_{sub}$ is the substrate bias voltage in V \\
$ C_{mis}$ is the MIS junction capacitance in $F/m^2$ \\
$ C_i$ is the insulator capacitance per unit area in $F/m^2$ \\
$ C_{depl}$ is the depletion layer in capacitance per unit area in $F/m^2$ \\
$ Q_s$ is the surface charge density in $C/m^2$ \\
$ V_g$ is the gate voltage in V \\
$ 2\phi_b$ is the change in the surface potential in V required to generate strong inversion \\
$ V_c(x)$ in V is the channel potential, which varies along the length of the channel in the \\
x direction \\
$ n_{ss}(x)$ is the effective electron density along the channel in $/m^2$ \\
$ \mu_n$ is the channel mobility in $m^2/V\cdot s$ \\
$ dV_c(x)/dx$ is the incremental change in the channel potential divided by \\
the incremental change in position in V/m \\
$ W_c$ is the channel width in m \\
$ V_{ds}$ is the drain-to-source voltage in V \\
L is the channel length in m
$ V_g-V_{ds}$ is the voltage drop across the drain-substrate PN junction in V \\
$ V_T$ is the threshold voltage in V \\
$ g_m$ is the transconductance of the MOSFET in $/\Omega or S$ \\
$ \gamma^{\prime}$ is an empirical parameter in /V \\
$ I_{D_{sat}}$ is the saturation current in A \\
$ I_{ds}$ is the drain-source current in A \\
$ G_v (=\Delta I_{ds}R_L/\Delta V_g)$ is the voltage gain \\
$ R_L$ is the resistive load in $\Omega$ \\
$ \Delta x$ is the incremental channel length in m \\
$ W_c$ is the width of the MESFET  in m \\
$ t_c$ is the thickness of the active layer in m \\
$ g_0$ is the channel conductance in $\Omega or S$ \\
$ V_{po} (=qN_Dt^2_c/(2\epsilon_s))$ is the pinch-off voltage in V \\
$ \gamma^{\prime}$ and $b$ are empirical constants in /V \\
$ I_{go}$ is the gate saturation current in A \\
$ R_g$ is the gate resistance in $\Omega$ \\
$ \eta_g$ is the ideality factor for the gate junction \\
$ D_{pb}$ is the hole diffusivity in the base in $m^2/s$ \\
$ p_{n0}$ is the equilibrium hole density in the base in $/m^3$ \\
$ \tau_p$ is the hole lifetime in s \\
$ A_{b1}$ and $A_{b2}$ are constants in $/m^3$ \\
$ L_{pb} (=\sqrt{D_{pb}/\tau_p}$ is the hole diffusion length in the base in m \\
$ L_{ne}$ and $L_{nc}$ in m are the electron diffusion lengths in the emitter and collector \\
$ x_e$ and $x_c$ in m are the widths of the base-emitter junc. and base-collector junc. \\
$ I_{ne}$ and $I_{nc}$ in A are the electron-diffusion currents in the emitter and collector \\
$ D_{ne}$ and $D_{nc}$ in $m^2/s$ are the diffusivities of the electrons in the \\
emitter and collector  \\
$ L_{ne}$ and $L_{nc}$ in m are the diffusion lengths of the electrons in the \\
emitter and collector \\
$ N_{Ae}$ is the acceptor density in the emitter in $/m^3$ \\
$ N_{Db}$ is the donor density in the base in $/m^3$ \\
$ Q_G$ is the Gummel number in base doping per unit area \\
$ \beta_g (=\Delta I_c/\Delta I_b)$ is the common-emitter current gain \\
$ $ \\
$ $ \\
$ $ \\
$ $ \\
$ $ \\
$ $ \\
$ $ \\
$ $ \\
$ $ \\
$ $ \\
$ $ \\
$ $ \\
$ $ \\
$ $ \\
$ $ \\
$ $ \\
$ $ \\
$ $ \\
$ $ \\
$ $ \\
\end{tabular}


\end{multicols}
\end{document}
